\newpage
\null\vfill
{\fontsize{7.5}{8}\selectfont
\begin{center}
\textbf{Génomique écologique de l'exploitation de niche et de la performance individuelle chez les arbres forestiers tropicaux} \\
\end{center} 
\textbf{Résumé :}
Les forêts tropicales abritent la plus grande diversité d'espèces au monde, un fait qui reste en partie inexpliqué et dont l'origine est sujette à débat. Même à l'échelle de l'hectare, les forêts tropicales abritent des genres riches en espèces, avec des espèces d’arbres étroitement apparentées qui coexistent en sympatrie. En raison de contraintes phylogénétiques, on s'attend à ce que les espèces étroitement apparentées possèdent des niches et des stratégies fonctionnelles similaires, ce qui questionne les mécanismes de leur coexistence locale. Les espèces étroitement apparentées peuvent former un complexe d'espèces, composé d’espèces morphologiquement similaires ou qui partagent une importante proportion de leur variabilité génétique en raison d'une ascendance commune récente ou d'hybridation, et qui peut résulter d'une radiation écologique adaptative des espèces selon des gradients environnementaux. Malgré le rôle clé des complexes d'espèces dans l'écologie, la diversification et l'évolution des forêts néotropicales, les forces éco-évolutives qui créent et maintiennent la diversité au sein des complexes d'espèces néotropicales restent peu connues. Nous avons exploré la variabilité génétique intraspécifique comme un continuum au sein de populations structurées d'espèces étroitement apparentées, et mesuré son rôle sur la performance individuelle des arbres à travers la croissance dans le temps, tout en tenant compte des effets d'un environnement finement caractérisé au niveau abiotique et biotique. En combinant des inventaires forestiers, des données topographiques, des traits fonctionnels foliaires et des données de capture de gènes dans la station de recherche de Paracou, en Guyane Française, nous avons utilisé la génomique des populations, les analyses d'associations environnementales et génomiques, et la modélisation Bayésienne sur les complexes d'espèces \emph{Symphonia} et \emph{Eschweilera} clade \emph{Parvifolia}. Nous avons montré que les complexes d'espèces d'arbres couvrent l’ensemble des gradients locaux de topographie et de compétition présents dans le site d'étude alors que la plupart des espèces qui les composent présentent une différenciation de niche marquée le long de ces mêmes gradients. Plus précisément, dans les complexes d'espèces étudiés, la diminution de la disponibilité en eau, par exemple depuis les bas-fonds jusqu’aux plateaux, a entraîné une modification des traits fonctionnels foliaires, depuis des stratégies d'acquisition à des stratégies conservatrices, tant entre les espèces qu'au sein de celles-ci. Les espèces de \emph{Symphonia} sont génétiquement adaptées à la distribution de l'eau et des nutriments, elles coexistent donc localement en exploitant un large gradient d'habitats locaux. Inversement, les espèces d'\emph{Eschweilera} sont différentiellement adaptées à la chimie du sol et évitent les habitats les plus humides et hydromorphes. Enfin, les génotypes individuels des espèces de \emph{Symphonia} sont différentiellement adaptés pour se régénérer et croître en réponse à la fine dynamique spatio-temporelle des trouées forestières, avec des stratégies adaptatives de croissance  divergentes le long des niches de succession. Par conséquent, la topographie et la dynamique des trouées forestières entraînent des adaptations spatio-temporelles à fine échelle des individus au sein et entre les espèces des complexes d'espèces \emph{Symphonia} et \emph{Eschweilera} clade \emph{Parvifolia}. Je suggère que les adaptations à la topographie et à la dynamique des trouées forestières favorisent la coexistence des individus au sein et entre les espèces des complexes d'espèces, et peut-être plus généralement entre les espèces d'arbres de forêts matures. Dans l'ensemble, je soutiens le rôle primordial des individus au sein des espèces dans la diversité des forêts tropicales, et suggère que nous devrions élaborer une théorie de l'écologie des communautés en commençant par les individus, car les interactions avec les environnements se produisent après tout au niveau de l’individu. \\

\textbf{Mots clés :}
Coexistence des espèces ; Complexe d'espèces ; Distribution des espèces ; Forêts tropicales ; Indice d'encombrement du voisinage ; Indice d'humiditétopographique ; Niche écologique ; Paracou ; Syngameon ; Variabilité intraspécifique \vspace*{\baselineskip}
\newline\noindent\rule{\textwidth}{0.7pt}

\begin{center}
\textbf{Ecological genomics of niche exploitation and individual performance in tropical forest trees} \\
\end{center} 
\textbf{Abstract:}
Tropical forests shelter the highest species diversity worldwide, a fact that remains partly unexplained and the origin of which is subject to debate. Even at the hectare-scale, tropical forests shelter species-rich genera with closely-related tree species coexisting in sympatry. Due to phylogenetic constraints, closely related species are expected to have similar niches and functional strategies, which raises questions on the mechanisms of their local coexistence. Closely related species may form a species complex, defined as morphologically similar species that share large amounts of genetic variation due to recent common ancestry and hybridization, and that can result from ecological adaptive radiation of species segregating along environmental gradients. Despite the key role of species complexes in Neotropical forest ecology, diversification, and evolution, little is known of the eco-evolutionary forces creating and maintaining diversity within Neotropical species complexes. We explored the intraspecific genomic variability as a continuum within structured populations of closely related species, and measured its role on individual tree performance through growth over time, while accounting for effects of a finely-characterized environment at the abiotic and biotic level. Combining tree inventories, LiDAR-derived topographic data, leaf functional traits, and gene capture data in the research station of Paracou, French Guiana, we used population genomics, environmental association analyses, genome-wide association studies and Bayesian modelling on the tree species complexes \emph{Symphonia} and \emph{Eschweilera} clade \emph{Parvifolia}. We showed that the species complexes of Neotropical trees cover all local gradients of topography and competition and are therefore widespread in the study site whereas most of the species within them exhibit pervasive niche differentiation along these same gradients. Specifically, in the species complexes \emph{Symphonia} and \emph{Eschweilera} clade \emph{Parvifolia}, the decrease in water availability due to higher topographic position, e.g., from bottomlands to plateaus, has led to a change in leaf functional traits from acquisitive strategies to conservative strategies, both among and within species. \emph{Symphonia} species are genetically adapted to the distribution of water and nutrients, hence they coexist locally through exploiting a broad gradient of local habitats. Conversely, \emph{Eschweilera} species are differentially adapted to soil chemistry and avoid the wettest, hydromorphic habitats. Last but not least, individual tree genotypes of \emph{Symphonia} species are differentially adapted to regenerate and thrive in response to the fine spatio-temporal dynamics of forest gaps with divergent adaptive growth strategies along successional niches. Consequently, topography and the dynamics of forest gaps drive fine-scale spatio-temporal adaptations of individuals within and among distinct but genetically connected species within the species complexes \emph{Symphonia} and \emph{Eschweilera} clade \emph{Parvifolia}. Fine-scale topography drives genetic divergence and niche differentiation with genetic adaptations among species, while forest gap dynamics maintains genetic diversity with divergent adaptive strategies within species. I suggest that adaptations of tree species and individuals to topography and dynamics of forest gaps promote coexistence within and among species within species complexes, and perhaps among mature forest tree species outside species complexes. Overall, I defend the primordial role of individuals within species in tropical forest diversity, suggesting that we should develop a theory of community ecology starting with individuals, because interactions with environments happen after all at the individual level. 

\textbf{Keywords:}
Ecological niche; Intraspecific variability; Neighbourhood crowding index; Paracou; Species coexistence; Species complex; Species distribution; Syngameon; Topographic wetness index;   Tropical forests

}
\vfill\null