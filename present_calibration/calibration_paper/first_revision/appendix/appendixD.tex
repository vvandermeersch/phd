% ARTICLE ----
% This is just here so I know exactly what I'm looking at in Rstudio when messing with stuff.
\documentclass[11pt,]{article}
\usepackage[left=1in,top=1in,right=1in,bottom=1in]{geometry}
\newcommand*{\authorfont}{\fontfamily{phv}\selectfont}
\usepackage[]{libertine}


  \usepackage[T1]{fontenc}
  \usepackage[utf8]{inputenc}




\usepackage{abstract}
\renewcommand{\abstractname}{}    % clear the title
\renewcommand{\absnamepos}{empty} % originally center

\renewenvironment{abstract}
 {{%
    \setlength{\leftmargin}{0mm}
    \setlength{\rightmargin}{\leftmargin}%
  }%
  \relax}
 {\endlist}

\makeatletter
\def\@maketitle{%
  \newpage
%  \null
%  \vskip 2em%
%  \begin{center}%
  \let \footnote \thanks
    {\fontsize{18}{20}\selectfont\raggedright  \setlength{\parindent}{0pt} \@title \par}%
}
%\fi
\makeatother


\renewcommand*\thetable{D.\arabic{table}}
\renewcommand*\thefigure{D.\arabic{figure}}


\setcounter{secnumdepth}{0}




\title{Estimating process-based model parameters from species
distribution data: Supplementary Appendix D  }



\author{\Large Victor Van der
Meersch\vspace{0.05in} \newline\newline\normalsize\emph{CEFE,
CNRS}   \and \Large Isabelle
Chuine\vspace{0.05in} \newline\newline\normalsize\emph{CEFE, CNRS}  }


\date{}

\usepackage{titlesec}

\titleformat*{\section}{\normalsize\bfseries}
\titleformat*{\subsection}{\normalsize\itshape}
\titleformat*{\subsubsection}{\normalsize\itshape}
\titleformat*{\paragraph}{\normalsize\itshape}
\titleformat*{\subparagraph}{\normalsize\itshape}





\newtheorem{hypothesis}{Hypothesis}
\usepackage{setspace}


% set default figure placement to htbp
\makeatletter
\def\fps@figure{htbp}
\makeatother

\usepackage{pdflscape}
\usepackage{xcolor}
\usepackage{graphicx}
\usepackage{float}
\floatplacement{figure}{H}
\usepackage[width=.9\textwidth, textfont=it, font=small]{caption}
\usepackage{libertine}
\usepackage{booktabs}
\usepackage{longtable}
\usepackage{array}
\usepackage{multirow}
\usepackage{wrapfig}
\usepackage{float}
\usepackage{colortbl}
\usepackage{pdflscape}
\usepackage{tabu}
\usepackage{threeparttable}
\usepackage{threeparttablex}
\usepackage[normalem]{ulem}
\usepackage{makecell}
\usepackage{xcolor}

% move the hyperref stuff down here, after header-includes, to allow for - \usepackage{hyperref}

\makeatletter

\makeatother


% Add an option for endnotes. -----


% add tightlist ----------
\providecommand{\tightlist}{%
\setlength{\itemsep}{0pt}\setlength{\parskip}{0pt}}

% add some other packages ----------

% \usepackage{multicol}
% This should regulate where figures float
% See: https://tex.stackexchange.com/questions/2275/keeping-tables-figures-close-to-where-they-are-mentioned
\usepackage[section]{placeins}

% CSL environment change -----


% Last minute stuff -----
% \usepackage{amssymb,amsmath} % HT @ashenkin


% Add by VV
% \usepackage[dvipsnames]{xcolor} % clash with kableExtra ?
\PassOptionsToPackage{hyphens}{url} % url is loaded by hyperref
\usepackage[unicode=true]{hyperref}
  \PassOptionsToPackage{usenames,dvipsnames}{color} % color is loaded by hyperref
\definecolor{um-red}{RGB}{233,78,82}
\definecolor{um-gray}{RGB}{114,146,162}
\definecolor{blue-sapphire}{RGB}{0, 110, 144}
\hypersetup{
    colorlinks=true,
    linkcolor=um-red,
    citecolor=um-red,
    urlcolor=um-gray,
    breaklinks=true}
\urlstyle{same} % don't use monospace font for urls
\usepackage[capitalise, nameinlink]{cleveref}




\begin{document}

% \pagenumbering{arabic}% resets `page` counter to 1
%%\renewcommand*{\thepage}{D--\arabic{page}}
%
% \maketitle

{% \usefont{T1}{pnc}{m}{n}
\setlength{\parindent}{0pt}
\thispagestyle{plain}
{\fontsize{18}{20}\selectfont\raggedright
\maketitle  % title \par

}

{
   \vskip 13.5pt\relax \normalsize\fontsize{11}{12}
\textbf{\authorfont Victor Van der
Meersch} \hskip 15pt \emph{\small CEFE,
CNRS}   \par \textbf{\authorfont Isabelle
Chuine} \hskip 15pt \emph{\small CEFE, CNRS}   

}

}






\vskip -8.5pt


 % removetitleabstract

\noindent 

\hfill \break

\hypertarget{box-constraint-handling}{%
\subsection{Box constraint handling}\label{box-constraint-handling}}

With this constraint handling - implemented by default in the R package
\emph{cmaes} {[}@Trautmann2011{]} - each evaluated solution is
guaranteed to lie within the feasible space. Let's say we have a
parameter vector \(x\). For each parameter \(x_i\), we have a lower
bound \(lb_i\) and an upper bound \(ub_i\). If a parameter \(x_i\)
violates one of this bound, we set \(x_i\) to a new value
\(x_i^{repaired}\) equal to the closest boundary value (\(lb_i\) or
\(ub_i\)). We thus obtained a new parameter set \(x^{repaired}\), with a
minimal \(\|x-x^{repaired}\|\) value. This new feasible solution
\(x^{repaired}\) is used for the evaluation of the objective function
\(AUC_{model}(x^{repaired})\), and to compute a penalty term
\(pen=\sum\limits_{i}(x_i-x_i^{repaired})^2=\|x^{repaired}-x\|^2\). Then
\(x^{repaired}\) is discarded, and the algorithm computes the penalized
objective function of \(x^{repaired}\) as follows:
\(AUC_{model}(x)=AUC_{model}(x^{repaired})+pen\). This boundary handling
could be improved with adaptive weights {[}see @Hansen2009{]}.

\hypertarget{ecological-infeasibility-constraint}{%
\subsection{Ecological infeasibility
constraint}\label{ecological-infeasibility-constraint}}

We added a simple way to handle ecological constraint (e.g.~unfolding
before flowering in beech mixed bud) with a death penalty. When a
parameter vector \(x\) violates a constraint, it is rejected and
generated again. The main drawback of this approach is that CMA-ES does
not use information from unfeasible points. An other approach could be
to set \(AUC_{model}(x)=0\). However, as our feasible space was large,
the death penalty constraint worked well in our case.

\textcolor{darkgreen}{*Fagus sylvatica* and *Quercus ilex* have mixed bud, so unfolding must happened before flowering. On the contrary, we did not applied inequality constraint on *Abies alba* simple bud phenology parameters.}





\newpage
\singlespacing
true

\end{document}
